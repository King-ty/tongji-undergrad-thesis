\section{讨论}\label{sec:discussion}

\subsection{相关工作}

MSL da Silva等人~\cite{}提出了Squad,通过一个SGX保护的数据存储节点,存储各个服务节点需要访问和使用的敏感数据,例如API密钥、证书、密码等等,同时可以对服务进行验证,由此实现对微服务应用的保护。但是Squad可能会成为整个系统的性能和安全瓶颈。同时,Squad与本文的设计并不是冲突的,Squad主要针对的是微服务应用启动和配置的问题,而本文主要为微服务提供运行时环境和通信的安全高效保障,因此可以配合使用共同致力于保护微服务的安全。

S Brenner等人~\cite{}提出了Vert.x Vault,通过改进Eclipse Vert.x框架,使其能够同时支持使用C/C++语言编写的安全微服务节点和使用Java语言编写的普通节点,从而实现了借助SGX保护微服务的目的。但是Vert.x Vault的实现受限于特点框架,且SGX中的节点仍需要使用C/C++语言重新编写,损失了灵活性且代价较高,且不能防止编程漏洞引入的攻击,因此安全性和适用性都不如本文的设计。

对于SGX中的运行时,成熟的库操作系统包括Graphene-SGX~\cite{}、Occlum~\cite{}等,而针对语言的运行环境包括Rust-SGX~\cite{}、GoTEE~\cite{}、Uranus~\cite{}、ScriptShield~\cite{}等,分别尝试在飞地中运行Rust、Go、Java和脚本语言等。这些运行时的设计各有特点,本文的设计也参考了其中的多种设计;但是这些设计本身不一定适用于微服务的安全计算场景(例如引入额外的TCB),因此本文的运行时设计仍然有其独特的价值。

\subsection{未来展望}

关于基于可信执行环境的安全高效微服务场景,本文的工作仍有一些局限性,因此我们认为未来的工作可以从以下几个方面进行:

\begin{itemize}
    \item \textbf{通过进一步的分析解决隐私扩散问题。}尽管本文中使用了基于Token的细粒度授权管理方案,引入了额外的开销,但是仍然不能完全解决隐私扩散问题,例如微服务节点B将节点A的数据泄露给C,但是A并不希望C能够获取这些数据(不同开发组之间的隐私协定),这一点无法使用Token授权完全避免。我们认为这是因为微服务节点间不能感知彼此的行为,即A不能得知和约束B对其发送过去的数据的处理方式。目前可以采用非技术的约束手段来减小这一问题,但是容易出现疏漏,因此我们需要进一步的研究,通过技术手段解决隐私扩散问题。
    \item \textbf{通过节点调度降低通信延迟。}微服务应用可以通过节点调度来降低通信延迟,例如将频繁通信的节点部署在同一个物理机上、同一个飞地中甚至是同一地址空间,这样可以大大减少通信延迟。但是这一点需要进一步的研究,达到在不降低安全性的前提下进行调度。
    \item \textbf{适配更多架构。}由于微服务应用具有异构性的特点,为了增强平台的适用性,我们需要进一步研究如何在不同的架构上实现本文的设计,达到类似的安全保障。
\end{itemize}
