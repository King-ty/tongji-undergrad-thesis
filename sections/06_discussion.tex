\section{讨论}\label{sec:discussion}

\subsection{相关工作}

MSL da Silva等人~\cite{}提出了Squad,通过一个SGX保护的数据存储节点,存储各个服务节点需要访问和使用的敏感数据,例如密钥、证书、配置信息等,同时可以对其他微服务进行验证,由此实现对微服务应用的保护。但是Squad节点可能会成为整个系统的性能和安全瓶颈。Squad与本文的设计并不是冲突的,Squad主要针对的是微服务应用启动和配置的问题,而本文主要为微服务提供了安全的运行时环境和安全高效的通信协议,保障了微服务运行时安全,因此可以配合使用共同致力于保护微服务的安全。

S Brenner等人~\cite{}提出了Vert.x Vault,通过改进Eclipse Vert.x框架,使其能够同时支持使用C/C++语言编写的安全微服务节点和使用Java语言编写的普通节点,从而实现了借助SGX保护微服务的目的。但是Vert.x Vault的实现受限于特定框架和语言,且SGX中的程序仍需要使用C/C++语言重新编写,损失了灵活性且代价较高,而且不能防止编程漏洞引入的攻击,因此安全性和适用性都不如本文的设计。

对于SGX中的运行时,成熟的库操作系统包括Graphene-SGX~\cite{}、Occlum~\cite{}等,而针对语言的运行环境包括Rust-SGX~\cite{}、GoTEE~\cite{}、Uranus~\cite{}、ScriptShield~\cite{}等,分别尝试在飞地中运行Rust、Go、Java和脚本语言等。这些运行时的设计各有特点,本文的设计也参考了其中的多种设计;但是这些设计本身不一定适用于微服务的安全计算场景(例如引入额外的TCB),因此本文的运行时设计仍然有其独特的价值。

\subsection{未来展望}

对于基于可信执行环境的安全计算微服务场景,本文的工作仍有一些局限性,因此我们认为未来的工作可以从以下几个方面进行:

\begin{itemize}
    \item \textbf{通过进一步的分析解决隐私扩散问题。}尽管本文中使用了基于Token交换的细粒度授权管理方案,引入了额外的开销,但是仍然不能完全解决隐私扩散问题,例如微服务节点B将节点A的数据泄露给C,但是A并不希望C能够获取这些数据(受限于不同开发组之间的隐私协定),这一点无法使用Token授权完全避免。我们认为这是因为微服务节点间不能感知彼此的行为,即A不能得知和约束B对其发送过去的数据的处理方式,B将数据传递给C这件事A不知情也无法阻止。目前可以采用非技术的约束手段来减小这一问题,但是这样容易出现疏漏,因此我们需要进行进一步的研究,通过技术手段解决隐私扩散问题。
    \item \textbf{通过节点调度降低通信延迟。}微服务应用可以通过节点调度来降低通信延迟,例如将频繁通信的节点部署在同一个物理机上、同一个飞地中甚至是同一地址空间,这样可以大大减少通信延迟。但是这一点需要进一步的研究,达到在不降低安全性的前提下进行调度。
    \item \textbf{适配更多架构。}由于微服务应用具有异构性的特点,为了增强平台的适用性,需要进一步研究如何在不同的架构上实现本文的设计,达到类似的安全保障。
\end{itemize}
