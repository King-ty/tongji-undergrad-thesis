\section{评估}\label{sec:evaluation}

\paragraph{实验环境}

我们在一台拥有Intel E3-1280 V6 CPU、64GB内存、支持SGX v1的服务器上运行和实验。该机器运行Ubuntu 22.04操作系统,内核版本为5.15.0-56-generic。

\subsection{性能表现}

在这部分,由于其他相关工作都没有开源实现,因此我们只是简单比较了微服务示例程序在SGX内外的通信和计算性能表现。如图XXX所示,我们可以看到,在SGX内外运行的微服务程序通信延迟差距并不大,在SGX中运行的微服务节点的通信带来了额外的XXX的延迟,但相比于其提供的安全保障,这点延迟是可以接受的。而在计算性能方面,由于SGX的特性,SGX内外的计算性能差距较大,但是相比于通信延迟,计算延迟是可以忽略不计的。

% TODO:要不要测试一下握手认证协议的性能?

\subsection{安全分析}

如~\ref{subsec:threat-model}讲述的威胁模型所示,本次设计主要针对的攻击手段包括系统特权态攻击、网络攻击、漏洞利用攻击等攻击手段,重点防御隐私泄露问题,结合信任模型,我们可以得到如下的安全性分析:

\begin{itemize}
    \item \textbf{系统特权态攻击}:因为硬件和SGX是可信的,因此根据SGX的防御模型,即使OS或HyperVisor是恶意的或被劫持,SGX中的数据也不会泄露,因此本平台可以防御系统特权态攻击。
    \item \textbf{中间人攻击}:对于中间人攻击,我们使用了DH密钥交换协议来保证通信的安全性,即使所有DH共享参数都被中间人截获,中间人也无法得到密钥,因此无法解读加密信息;同时通过使用SGX远程验证,可以防止中间人伪造节点进行攻击。
    \item \textbf{重放攻击}:对于重放攻击,我们在通信协议中加入了计数器,且借助SGX和微服务计算的场景,消除了TLS 0-RTT状态恢复情况的漏洞,可以成功过滤重放信息。
    \item \textbf{编程漏洞带来的攻击}:对于系统编程语言容易出现的悬垂指针、缓冲区溢出等漏洞带来的攻击,我们选择使用安全编程语言Rust,结合其强大的静态分析能力,基本可以消除这些情况(除非使用unsafe)。
    \item \textbf{权限扩散}:我们使用JWT的方法进行授权管理,类似于Capability的思想,因此为了防止权限扩散到下级节点,我们使用了Token交换的授权管理方法,可以实现细粒度的授权管理,减少隐私扩散。
\end{itemize}

另外,在一个完善的微服务场景中,还可能存在物理攻击、DDoS攻击、针对数据库的攻击等,但是针对这些攻击的应对措施与本次研究是正交的领域,因此不在本次研究的讨论范围内。
