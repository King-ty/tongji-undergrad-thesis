\section{引言}\label{sec:introduction}
随着面向服务架构(SOA)的发展,微服务~\cite{}架构已经成为了一种日渐流行的架构风格,并得到了工业界的广泛采用~\cite{}。与传统的单一服务(monolithic service)模式不同,微服务架构将一个庞大的服务拆分为多个独立的、简单的、解耦合的微服务,每个微服务都可以独立部署、升级、扩展,微服务之间通过网络进行通信,因此可以突破传统软件架构的限制,大大提升软件开发的灵活性和效率。

% 微服务架构的典型应用场景包括:Netflix的视频点播服务~\cite{}、Amazon的电子商务服务~\cite{}、Uber的叫车服务~\cite{}等等。

但是,在提供了开发、部署等方面的灵活性和便利性的同时,微服务架构也带来了一些新的挑战,并引入了许多新的攻击面。

首先是运行环境的安全问题。平台即服务(PaaS)是一种云计算模型,提供了可用于开发、运行和管理应用程序的完整的云平台(包括硬件、软件和基础设施),能够为服务提供自动化扩展等能力,因此微服务通常部署在PaaS上。然而,使用公有云的微服务需要信任公有云,以保证其不会窃取用户的私密信息;如果公有云平台是恶意的或者被恶意攻击者劫持,很容易威胁到其上运行的微服务的安全性。因此,公有云极大地限制了公有云微服务的使用场景(如人脸数据处理等)。

然后是微服务节点间的通信问题。由于分布式、解耦合的特点,微服务节点之间需要频繁的通信,这在单一服务的基础上额外引入了通信攻击面,对微服务架构的通信安全和节点认证等方面提出了挑战。例如,微服务节点之间的通信可能会被窃听、篡改,从而导致数据泄露、数据篡改等问题;恶意的或被劫持的节点可能会骗取正常节点的信任,从而窃取正常节点的机密信息,造成隐私泄露。同时,大量微服务节点的多级通信关系也对通信的效率提出了挑战,研究表明,传统的单一服务只需要满足百毫秒级别的延迟要求即可对用户提供高质量的服务,而微服务场景下节点间的通信需要慢走亚毫秒(sub-ms)的要求,才能在总体延迟上达到和单一服务相同的效果~\cite{}。

为了解决运行环境的安全问题,我们可以借助可信执行环境(TEE)提供的保障。可信执行环境是一种硬件级别的安全容器,可以对运行在其内部的代码、数据提供机密性、完整性的支持,从而保护微服务应用的安全。可信执行环境的典型实现包括Intel的SGX~\cite{}、ARM的TrustZone~\cite{}、AMD的SEV~\cite{}以及RISC-V上开源的Keystone~\cite{}等。因为可信执行环境结合了密码学原理和硬件级别的保护,可以在不信任云上系统的情况下保护数据的机密性和完整性,因此可以基本排除云上的安全威胁。目前的主流公有云大多是Intel架构的,因此我们选择使用SGX来提供保护;但是由于SGX主要是提供了硬件级别的隔离,只有基本的库函数而不能提供POSIX这样的接口,不能直接移植现有的程序,因此我们需要在SGX的基础上针对微服务的特点设计合适的运行时系统,以支持微服务的运行。

目前SGX上的运行环境主要有两种设计思路。第一种是在SGX中运行一个小型的操作系统,比如Microkernel,例如Keystone的XXX~\cite{},或LibOS,例如Graphene-SGX~\cite{}、XXX~\cite{}等,这样可以在无改动或极少改动的情况下将现有程序运行在SGX中;但是即使是LibOS也包含较多的代码,具有较大的可信信任基(TCB),可能存在较多漏洞和较大的攻击面。第二种思路是针对特定编程语言编写SGX内的运行时库,例如Rust-SGX~\cite{}、(SGX中的libc)~\cite{}等,这样可以达到很小的TCB;但是这种方式只能针对特定的编程语言,且不能实现所有的系统功能。在本课题中,根据微服务应用的特点,我们设计了一种将二者结合的运行时实现方式,能够实现在较小TCB的情况下为应用提供SGX的保护。

为了解决通信的问题,我们

另外,为了防止编程bug,我们选择使用安全编程语言进行保障。

% TODO:单一语言,没有则Rust-SGX,否则缩减Gramine,封装成Unikernel,达到较小的TCB。
%内部运行一个完整的操作系统,如Graphene~\cite{}、SCONE~\cite{}等;第二种是在SGX内部运行一个轻量级的运行时系统,如Occlum~\cite{}、Graphene-SGX~\cite{}等。第一种设计思路的优点是可以直接运行现有的应用程序,但是由于操作系统的复杂性,其本身的代码量较大,因此需要信任较多的代码,同时由于操作系统的复杂性,其本身的安全性也难以保证。第二种设计思路的优点是可以在SGX内部运行一个轻量级的运行时系统,因此可以大大减少运行时系统的代码量,从而减少信任的代码量,同时也可以提高运行时系统的安全性。但是,由于SGX的特性,其内部的运行时系统只能运行在SGX的受保护内存中,因此无法直接运行现有的应用程序,需要对应用程序进行修改,从而增加了开发的成本。


。其中,为提高云服务的可用性和降低维护成本,微服务(MicroService)将一个庞大服务(monolithic service)通过独立的微型服务的组合来实现;而微型服务间通过远端过程调用(RPC)进行通信。然而,使用公有云的微服务需要信任公有云,以保证其不会窃取用户的私密信息,极大地限制了公有云微服务的使用场景(如人脸数据处理等)。
可信执行环境(如SGX和TrustZone)可以对运行在可信执行环境的的数据、代码提供机密性、完整性的支持。因此,可信执行环境可以适用于微服务来保护处理敏感数据的微服务。
然而,尽管微服务计算可以降低开发的成本,微服务计算本身带来的信任问题为其实现带来了很大的挑战。首先,云应用程序需要与大量的系统组件一起运行,具有较大的TCB。然而由于其本身较大的代码量与较高的复杂性,他们很可能具有较大的漏洞,而这会被攻击者利用并带来内存安全上的挑战;另一方面,云服务商本身是不可信的,运行在其上的恶意程序可能会读取、篡改用户运行时数据,或对微服务间的通信进行篡改。这些都对对微服务本身的机密性和完整性提出挑战。
本课题旨在开发一个基于可信执行环境的安全高效微服务计算平台。首先本课题需要基于现有的微服务计算本身的特点,设计细粒度、模块化的运行时系统来减少TCB;其次本课题需要设计针对微服务计算的远程验证协议;最后本课题需要将其高性能地实现在可信执行环境中并测量不同微服务的性能。

