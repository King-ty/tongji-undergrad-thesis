\section{实现}\label{sec:implimentation}

我们简单实现了\ref{sec:overview}中所述的设计,受限于工程能力和时间,我们对一些不重要但困难的部分进行了简化,重点是使实现的平台能够模拟出一个简单的接近真实的场景,以此说明设计的安全性和合理性,并对性能进行初步的测试。

\paragraph{运行时实现}

由于本次设计的微服务示例程序仅用到了网络通信这一种操作系统的支持,因此我们在Graphene-SGX的基础上删掉了大部分操作系统的特性,只保留了能够支持微服务运行的最小功能,包括网络通信、内存管理、远程验证等,由此作为运行时实现。
% 我们将这个运行时实现命名为Graphene-SGX-Micro,它的代码量约为Graphene-SGX的1/10。

\paragraph{远程验证协议实现}

SGX远程验证是Graphene-SGX中已实现的功能,因此不用进行实现。

而安全通信协议由于是本次研究中针对微服务安全计算场景设计的,因此我们需要自己实现。我们按照之前设计的流程,使用Rust crate进行了简单的实现,并封装为库的形式,供程序进行链接与使用。因为实现通信协议库需要考虑很多细节,而本次只进行了简单的实现,因此不能保证实现安全,不能用于生产场景。

\paragraph{微服务实现}

我们按照\ref{sec:overview}中的微服务设计,使用Rust Tonic~\cite{}库实现了除数据持久化节点之外的所有微服务节点,并使用jsonwebtoken~\cite{}库生成和验证JWT,用来实现基于Token交换的授权机制。持久化节点暂时使用Redis进行模拟。
