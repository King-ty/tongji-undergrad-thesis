\section{平台实现}\label{sec:implimentation}

我们简单实现了\cref{sec:overview}中所述的设计,受限于工程能力和时间,我们对一些不重要但比较复杂的部分进行了简化,重点是使实现的平台能够表达出一个简单但接近真实的场景,以此说明设计的安全性和合理性,并对性能进行初步的测试。

\paragraph{运行时实现}

由于本次设计的微服务示例程序用到的操作系统支持很少,因此我们在Graphene-SGX~\cite{tsai2017graphene}的基础上删掉了大部分操作系统的特性,只保留了能够支持微服务运行的最小功能,包括网络通信、内存管理、远程验证等,由此作为运行时实现。
% 我们将这个运行时实现命名为Graphene-SGX-Micro,它的代码量约为Graphene-SGX的1/10。

\paragraph{远程验证协议实现}

向SGX IAS请求远程验证是Graphene-SGX中已实现的功能,因此不用进行实现。

由于安全通信协议的流程是本次研究中针对微服务安全计算场景设计的,因此我们需要自己实现。我们按照协议中设计的流程,简单实现了这一协议,实现形式为一个Rust crate,并将其封装为库的形式,供程序进行链接与使用。在通信的检查部分,为了方便实现,2层检查并没有采用并行的方式,而是按照顺序依次进行。对于信任白名单T,我们在哈希Map的基础上实现;为防止其他微服务地址变化导致表中记录弃用,占用过多内存,我们适当限制了T的大小,并使用最近最少使用(LRU)算法清理弃用项。
% 因为实现一个完整的通信协议库需要考虑很多细节(例如OpenSSL~\cite{openssl}),而本次实现不着重讨论这些细节,因此不能保证库的实现是安全的,不能用于生产场景。

\paragraph{微服务示例实现}

我们按照\cref{subsec:ms-example}中的微服务设计,使用Rust的gRPC库Tonic~\cite{tonic}实现了除数据持久化节点之外的所有微服务节点,并使用jsonwebtoken库生成和验证JWT,用来实现基于Token交换的授权机制。数据持久化节点暂时使用开源的Redis~\cite{redis}进行模拟。

为了进行测试,我们还实现了一个简单的客户端程序,用来模拟用户的请求。为简化处理,客户端和API网关之间的通信也使用了gRPC协议,而不是客户端和服务端交互下更常用的HTTP协议。
